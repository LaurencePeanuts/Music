
% !TEX TS-program = pdflatexmk
\documentclass[prd, onecolumn, floatfix, letterpaper, nofootinbib, amsmath, amssymb, superscriptaddress]{revtex4}
\usepackage{caption}
%\usepackage{subcaption}
\usepackage{graphicx}
\usepackage{epsfig}
\usepackage{bm}
\usepackage{amsfonts}
\usepackage{comment}
\usepackage{tikz}
\usetikzlibrary{shapes,arrows}

\usepackage{color}% use if color is used in text
\newcommand{\tc}{\textcolor{red}}

\newcommand{\ha}{{1 \over 2}}
\newcommand{\la}{\lambda}
\newcommand{\rc}{\nonumber\\}

\newcommand{\rhophi}{\rho_{\rm DDE}}
\newcommand{\de}{\partial}
\newbox\pippobox

\def\bx{\bm{x}}
\def\bk{\bm{k}}
\def\tpk{{\tilde{\phi}_k}}

%\newcommand{\eqn}[1] {Eq.~(\ref{#1})}
%\newcommand{\fig}[1] {Fig.~(\ref{#1})}
\newcommand {\lla} {\ {\raise-.5ex\hbox{$\buildrel<\over\sim$}}\ }
\renewcommand{\(}{\left(}
\renewcommand{\)}{\right)}
\renewcommand{\[}{\left[}
\renewcommand{\]}{\right]}

\def\iPP{\color{red}}
\def\bl{\color{green}}

\usepackage[countmax]{subfloat}
\usepackage[T1]{fontenc}
\usepackage[latin1]{inputenc}
\usepackage{graphicx}
\usepackage[english]{babel}
\usepackage{amsmath}
\usepackage{amssymb}
\usepackage{amsfonts}

\def\pp{{\, \mid \hskip -1.5mm =}}
\def\cL{{\cal L}}
\def\tr{{\rm tr}\, }
\def\nn{\nonumber \\}
\def\e{\mathrm{e}}

\def\be{\begin{equation}}
\def\ee{\end{equation}}
\def\bea{\begin{eqnarray}}
\def\eea{\end{eqnarray}}
\newcommand{\ex}{\mathrm{e}}
\newcommand{\dd}{\mathrm{d}}
\newcommand{\gsim}{\gtrsim}
\newcommand{\lsim}{\lesssim}
\def\spose#1{\hbox to 0pt{#1\hss}}
\def\Vec#1{\mbox{\boldmath$#1$\unboldmath}}
\def\lta{\mathrel{\spose{\lower 3pt\hbox{$\mathchar"218$}}
     \raise 2.0pt\hbox{$\mathchar"13C$}}}
\def\gta{\mathrel{\spose{\lower 3pt\hbox{$\mathchar"218$}}
     \raise 2.0pt\hbox{$\mathchar"13E$}}}
\newcommand{\trace}{\mathrm{Tr}}
\def\setH{\mathbb{H}}
\def\setR{\mathbb{R}}
\def\setC{\mathbb{C}}
\def\setG{\mathbb{G}}
\def\setO{\mathbb{O}}
\def\setN{\mathbb{N}}
\def\setZ{\mathbb{Z}}
\def\setUni{\mathbb{1}}

% Cosmology
\newcommand{\Hu}{\mathcal{H}}
\newcommand{\Ka}{\mathcal{K}}
\newcommand{\Zed}{\mathcal{Z}}
\newcommand{\cs}{c_{_\mathrm{S}}}
\newcommand{\ns}{n_{_\mathrm{S}}}

% General Physics
\newcommand{\GN}{G_{_\mathrm{N}}}
\newcommand{\mP}{m_{_\mathrm{Pl}}}
\newcommand{\Mp}{M_{_\mathrm{Pl}}}
\newcommand{\lP}{\ell_{_\mathrm{Pl}}}
\newcommand{\Lag}{\mathcal{L}}



\begin{document}


\title{Music of the Sphere: Connecting the Evolution of the Universe on the Largest Scales to Inflation}


\author{ Laurence Perreault Levasseur}
\maketitle



\section{Definitions and Constants}

Inflation provides a way of seeding perturbations to the metric outside the Hubble horizon (well) before the time of BBN. It is common to quantify these perturbations with the gauge-invariant curvature perturbation of the metric $\zeta$, which is defined by:
\be
	-\zeta=\psi+H\frac{\delta\rho}{\rho}\, ,
\ee
where $\psi$ is the trace part of the spatial scalar metric perturbations, i.e. writing the most general spatial perturbation to a 4-d metric by $\delta g_{ij}=2(\psi\delta_{ij}-E_{ij})$ with $\nabla^2E=0$, then $\psi$ contains the trace of the perturbation. We also define $\rho$ and $\delta \rho$ to be the mean energy density and the linear energy density perturbation, respectively, as well as the $H$, the Hubble parameter. $H$ and $\rho$ are related through the first Friedmann equation,
\be	
\label{eq:Hubbledef}
	H^2\equiv \left(\frac{\dot{a}}{a}\right)^2=\frac{1}{3\Mp}\rho=\frac{1}{3\Mp}\left( \frac{\dot\varphi^2}{2}+\frac{1}{2}\frac{\nabla^2\varphi}{a^2}+V \right)\, ,
\ee
 where a is the scale factor and, in the third equality, we have related $\rho$ to the field content of the Universe assuming a single scalar field, $\varphi$, is dominating its energy density, as is the case in single field inflation. The 3 terms in the l.h.s. of (\ref{eq:Hubbledef}) represent, in order of appearance, the kinetic energy density of the field, its gradient energy density, and its potential energy. 
 
 For inflation to occur, the field $\varphi$ must be dominated by its homogeneous mode, hence at the background level one has $\frac{1}{2}\frac{\nabla^2\varphi}{a^2}=0$, and its potential energy must dominate over its kinetic energy density. Hence $\dot\varphi^2\ll V$ during inflation. Moreover, during single-field inflation, there is only one physical scalar perturbation degree of freedom representing both scalar metric fluctuations and $\varphi$ fluctuations. Therefore, through an appropriate gauge choice, it is possible to gauge away all linear fluctuations in $\varphi$ and be left with only $\zeta$ as the scalar perturbation. This gauge choice is commonly called the $\zeta$-gauge, or comoving gauge.

During inflation, $H$ remains almost constant, since $V\gg\dot\varphi^2$ and $\varphi$ is almost static. Deviations from $H=\,$cst can be quantified by a hierarchy of slow-roll parameters:
\be
	\epsilon\equiv-\frac{\dot H}{H^2}\, ;\quad \eta=\frac{\dot \epsilon}{\epsilon H}\, ; \quad \eta_n=\frac{\dot{\eta}_{n-1}}{\eta_{n-1}N} ~~\mathrm{for~}n>1\, .
\ee
Inflation requires $\epsilon<1$, and obtaining at least 60 $e$-folds requires $\eta<1$ as well. In the case of single-field inflation, which we will assume in the rest of these notes, the first slow-roll parameter $\epsilon$ can also be re-written as:
\be
	\epsilon=\frac{1}{2}\frac{\dot\varphi^2}{\Mp^2H^2}\, ,
\ee
a formula that will be very useful in the following. Moreover, it will prove useful to use this to re-write $H$ as a function of $\epsilon$ and $V$ only:
\be
\label{eq:HnVrel}
	\Mp^2H^2=\frac{V}{3-\epsilon}
\ee

\bigskip 
The main observable predicted by inflation is the power spectrum of curvature fluctuations, $\mathcal{P}_\zeta (k)$, defined by:
\be
	\langle \zeta_k \zeta_k'\ \rangle=(2\pi)^{(3)}\delta^3(k+k')\mathcal{P}_\zeta(k)
\ee
Explicitly, $\mathcal{P}_\zeta (k)$ can be calculated to be
\be
	\mathcal{P}_\zeta(k)=\left.\frac{1}{4 k^3}\frac{H^2}{\epsilon}\right|_{k=aH}\, .
\ee
Quantities in the above equation are evaluated at horizon exit for every mode. Since each mode $\zeta_k$ remains constant outside of the Hubble horizon (see, e.g. Weinberg's proof in astro-ph/0302326), it is sufficient to calculate the amplitude of each mode at horizon exit and fix it from there until horizon re-entry, much after inflation.  If we do not want to perform the evaluation at horizon crossing, it is also possible to write an expression involving the explicit $k$-dependence, through a Hankel function of the first kind:
\be
	\mathcal{P}_\zeta(k)\sim \frac{\pi}{2}(-\tau)^{2\nu}\left| H^{(1)}_\nu(-k\tau)\right|^2\, .
\ee
Here, $\tau$ is the conformal time, related to the cosmic time $t$ by $dt=ad\tau$, and $\nu$ is given by:
\be
	\nu=\frac{3}{2}+\epsilon+\frac{1}{2}\eta
\ee
to leading order in slow-roll parameters. 
The dimensionless form of the power spectrum is given by
\be
	\Delta^2_\xi\equiv\frac{k^3}{2\pi^2}\mathcal{P}_\zeta(k)\,.
\ee


Other useful observables to define are the tilt of the power spectrum, defined by:
\be
	n_s-1\equiv\frac{d\ln\Delta^2_\zeta}{d\ln k}=-\epsilon-2\eta\, ,
\ee
where the last equality holds to first order in slow-roll parameters, and the running, defined by (to leading order in slow roll)
\be
	\alpha_s=-2\eta\epsilon-\eta\eta_2\, .
\ee
Finally, the tensor-to-scalar ratio is given by:
\be
	r=16\epsilon\, .
\ee
The amplitude of the power spectrum is usually given at a pivot scale, denoted by $k_*$. For $k_*=0.05Mpc^{-1}$, we have $\Delta^2_\zeta=2.4\times10^{-9}$ from the latest Planck data.

\section{Relation to the CMB}

The Fourier modes of $\zeta$ are related to observable coefficients of spherical harmonics on the last scattering surface through:
\be
	a_{lm}=4\pi (-i)^l\int \frac{\mathrm{d}^3k}{(2\pi)^3} \Delta_{T(l)}(k)\zeta_{\vec{k}}Y_{l,m}(\hat{k})\, ,
\ee
where the  $Y_{l,m}(\hat{k})$s are the spherical harmonics and $\Delta_{T(l)}(k)$ is a transfer function that transforms the primordial, 3 dimensional $\zeta_{\vec{k}}$ at horizon re-entry into the spectrum of metric fluctuations as observed today projected onto the 2d sphere, and can be written as
\be
	\Delta_{T(l)}(k)=\int_0^{\tau_0} d\tau S_T(k, \tau)P_{T, l}(k\left|\tau_0-\tau\right|)\,.
\ee 
This can be understood as an integral over the line of sight over a physical source term ($S_T$) and a geometric projection $P_{T,l}$ which can be written as a combination of Bessel functions.

Note that, inside the horizon after indlation, it is more convenient to use the Newtonian gauge. Once modes have re-entered the horizon, one can think of $\zeta$ and the gravitational potential $\phi$ (i.e. the metric perturbation in Newtonian gauge) interchangeably.

Moreover, upon horizon re-entry, all the modes $\zeta_{\vec{k}}$ with a fixed $|\vec{k}|$ are in phase and have an amplitude that is randomly distributed. If the fluctuations are Gaussian, the modes  $\zeta_{\vec{k}}$ with a fixed $|\vec{k}|$ are drawn from a Gaussian disrtibution with mean 0 and variance given by $P_\zeta(k)$. 

The reconstruction presented in music.pdf allows to recover $\zeta_k$ (or equivalently $\left| \zeta_k\right|^2$ which is a sample of $\mathcal{P}_\zeta(k)$) for a small, low value range of $k$-modes.  
 
 
 
 \section{Measurements of $|\zeta_k|^2$}
 
 
 Our goal in the following will be to reconstruct the potential of the inflaton in a model-independent fashion, by using only the samples of the specific $\zeta_k$ we get from reconstructing the fluctuations of the gravitational potential in the 3d Universe. Since in $k$-space we now look at a volume, the number for modes will therefore grow as $k^3$, as opposed to the usual $l^2$ when the Universe is projected onto a 2d surface at last scattering. It is useful to see how many samples we get for each additional surface $k$-shell we add to the considered volume. That is, for every additional shell in $\vec{n}$ space of unit thickness we add the number of additional samples of the norm of $\zeta_k$ is
 \bea
 	\delta Vol_{\vec{n}}&=&\frac{4\pi}{3}\left(n_2^3-n_1^3 \right)\\
	&=&\frac{4\pi}{3}\left(n_2^3-(n_2-1)^3 \right)\\
	&=&4\pi(n^2-n+1)\, .
 \eea
Therefore, for every additional shell, we add $2\pi(n^2-n+1)$ independent measurements, where we divide by two to only consider the half shell, since $\zeta_k$ and $\zeta_{-k}$ are related to each other through complex conjugation and are therefore not independent. Therefore the procedure from Music.pdf gives us a number of fiducial measurements of $|\zeta(k)|^2$ over a range of $|\vec{k}|$ (or $|\vec{n}|$) values.
 
 
 \section{Reconstructing the Potential of the Inflaton}
 
 Only assuming single field inflation, from the measurements of $|\zeta_k|^2$ it is possible to reconstruct locally the shape on the inflaton potential, in a model-independent way. To see how this is possible, we can re-write the expression of the $|\zeta_k|^2$ in terms of $V(\varphi)$ and slow-roll parameters, using (\ref{eq:HnVrel}): 
\be
\label{eq:PfctVnepsilon}
	|\zeta_k|^2=\mathcal{P}_\zeta(k)=\left.\frac{1}{4k^{3}}\frac{V}{\Mp^2\epsilon(3-\epsilon)}\right|_{k=aH}\, .
\ee
 
Given a single fiducial measurement of $\mathcal{P}_\zeta(k)$ at a fiducial value of $k_*$, we can find an expression for $V(\varphi)$ in the local neighbourhood of $V(\varphi)$ as a function of $V(\varphi_*)$ and slow-roll parameters at $\varphi_*$, where $\varphi_*$ is the value of the background inflaton field at the moment when the mode with wavenumber $k_*$ exited the horizon. To do this, we simply Taylor expand $V$ around $\varphi_*$:
\begin{align}
	V(\varphi_*+\Delta\varphi)&=V(\varphi_*)+\left.\partial_\varphi V(\varphi)\right|_{\varphi_*}\Delta\varphi+\left.\partial^2_\varphi V(\varphi)\right|_{\varphi_*}\frac{\Delta\varphi^2}{2}+\left.\partial^3_\varphi V(\varphi)\right|_{\varphi_*}\frac{\Delta\varphi^3}{3!}+\left.\partial^4_\varphi V(\varphi)\right|_{\varphi_*}\frac{\Delta\varphi^4}{4!}+...\, ;\\
	&=V(\varphi_*)\left[1+\frac{\Mp}{V(\varphi_*)}\left.\partial_\varphi V(\varphi)\right|_{\varphi_*}\frac{\Delta\varphi}{\Mp}+\frac{1}{2}\frac{\Mp^2}{V(\varphi_*)}\left.\partial^2_\varphi V(\varphi)\right|_{\varphi_*}\frac{\Delta\varphi^2}{\Mp^2}+\frac{1}{3!}\frac{\Mp^3}{V(\varphi_*)}\left.\partial^3_\varphi V(\varphi)\right|_{\varphi_*}\frac{\Delta\varphi^3}{\Mp^3}\right. \nonumber\\
	&\qquad\qquad\qquad\left.+\frac{1}{4!}\frac{\Mp^4}{V(\varphi_*)}\left.\partial^4_\varphi V(\varphi)\right|_{\varphi_*}\frac{\Delta\varphi^4}{\Mp^4}+...\right]\, ;\\
	&\equiv V(\varphi_*)\left[1+d_1\frac{\Delta\varphi}{\Mp}+\frac{1}{2}d_2\frac{\Delta\varphi^2}{\Mp^2}+\frac{1}{3!}d_3\frac{\Delta\varphi^3}{\Mp^3} +\frac{1}{4!}d_4\frac{\Delta\varphi^4}{\Mp^4}+...\right]\,.
\end{align}

where we have included only relevant and marginal operators in the expansion (up to the fourth derivative of $V$). Higher derivatives represent irrelevant operators and are neglected here. Since we are only performing a local expansion around a point of the potential for which the range of the inflaton is small and are not considering the full range of the inflaton during inflation, we do not expect a breakdown of perturbation theory; i.e., we consider sub-Planckian excursions of the field around $\varphi_*$, $\Delta\varphi<\Mp$, so that the tower of higher order operators suppressed by at least 5 powers of the Planck mass do not become relevant.

Similarly, we can expand $\epsilon$ appearing in (\ref{eq:PfctVnepsilon}) around the fiducial value $\epsilon_*$ at the same $\varphi_*$,
\be
	\epsilon=\epsilon_*+\Gamma_1\frac{\Delta\varphi}{\Mp}+\frac{\Gamma_2}{2}\frac{\Delta\varphi^2}{\Mp^2}+\frac{\Gamma_3}{3!}\frac{\Delta\varphi^3}{\Mp^3}+\frac{\Gamma_4}{4!}\frac{\Delta\varphi^4}{\Mp^4}\, .
\ee
 
Every derivative of $V$ (more specifically, the $d_i$ parameters) can be expressed in terms of slow-roll parameters at $\varphi_*$ alone, and similarly for all the derivatives of $\epsilon$. Therefore, we find a fitting function in terms of the slow-roll parameters. The explicit expressions for the $\Gamma_i$ and $d_i$ were obtained in the calculation notes accompanying this file. The $\Gamma_i$ parameters are given by:
\begin{align}
	\Gamma_1&=\sqrt{\frac{\epsilon}{2}}\eta\,;\\
	\Gamma_2&=\frac{\eta\eta_2}{2}+\frac{\eta^2}{4}\,;\\
	\Gamma_3&=\frac{1}{2\sqrt{2\epsilon}}\left[ \eta_3\eta_2\eta+\eta^2\eta_2+\eta_2^2\eta \right]\, ;\\
	\Gamma_4&= \frac{\eta\eta_2}{4\epsilon}\left[ \eta_4\eta_3+\eta_3^2+3\eta_3\eta_2+\eta_2^2+\frac{3}{2}\eta\eta_2 -\frac{1}{2}\eta^2+\frac{1}{2}\eta\eta_3 \right]\, .
\end{align}
The $d_i$ parameters are given by:
\begin{align}
\label{eq:Gamma1}
	d_1&=\sqrt{2\epsilon}{(3-\epsilon)}\left[ -3+\epsilon-\frac{1}{2}\eta \right]\, ,\nonumber\\
		& \approx -\sqrt{2\epsilon}\left(1+\frac{1}{6}\epsilon+\frac{1}{18}\eta\epsilon \right)\, ;\\
	d_2&= -\frac{1}{12}\frac{1}{(3-\epsilon)}\left(-24\epsilon+6\eta+2\eta\eta_2+8\epsilon^2 -10\epsilon\eta+\eta^2\right)\, ,\nonumber\\
		&\approx 2\epsilon-\frac{1}{2}\eta-\frac{1}{6}\eta\eta_2\, ;\\
	d_3&=\, \frac{1}{2\sqrt{2\epsilon}} \frac{1}{(3-\epsilon)}\left[ -24\epsilon^2 +18\epsilon\eta +7\epsilon\eta\eta_2+8\epsilon^3-18\epsilon^2\eta+6\epsilon\eta^2-\eta\eta_2\left(3+\eta+\eta_2+\eta_3 \right)\right] ,\nonumber\\
		&\approx \frac{8\epsilon^2-6\epsilon\eta-\frac{8}{3}\epsilon\eta\eta_2+\eta\eta_2\left(1+\frac{\eta+\eta_2+\eta_3}{3} \right)}{2\sqrt{2\epsilon}}\\
	d_4&=\frac{1}{(3-\epsilon)}\left[12\epsilon^2-18\epsilon\eta+\frac{9}{4}\eta^2 +6\eta\eta_2-4\epsilon^3+14\epsilon^2\eta-\frac{39}{4}\epsilon\eta^2   +\frac{3}{4}\eta^3 -8\epsilon\eta\eta_2+\frac{35}{8}\eta^2\eta_2+\frac{9}{4}\eta\eta_2^2+\frac{9}{4}\eta\eta_2\eta_3\right.\nonumber\\
	& \qquad +\frac{1}{\epsilon}\left.\left( \frac{3}{8}\eta^2\eta_2-\frac{3}{4}\eta\eta^2_2-\frac{3}{4}\eta\eta_2\eta_3 +\frac{\eta^3\eta_2}{8}-\frac{3}{8}\eta^2\eta_2^2-\frac{1}{4}\eta\eta_2^3-\frac{3}{4}\eta\eta_2^2\eta_3-\frac{1}{8}\eta^2\eta_2\eta_3-\frac{1}{4}\eta\eta_2\eta_3^2-\frac{1}{4}\eta\eta_2\eta_3\eta_4\right)\right]
	\, , \\
		&\approx 4\epsilon^2-6\epsilon\eta+\frac{3}{4}\eta^2+\frac{4}{3}\eta\eta_2+
		\frac{\eta\eta_2}{3\epsilon}\left( \frac{3}{8}\eta-\frac{3}{4}\eta_2-\frac{3}{4}\eta_3 +\frac{\eta^2}{8}-\frac{3}{8}\eta\eta_2-\frac{1}{4}\eta_2^2-\frac{3}{4}\eta_2\eta_3-\frac{1}{8}\eta\eta_3-\frac{1}{4}\eta_3^2-\frac{1}{4}\eta_3\eta_4\right)\nonumber\\
		&\qquad+\frac{\eta\eta_2}{3}\left[ 8\epsilon+\frac{9}{2}\eta+2\eta_2+2\eta_3+\frac{3}{4}\epsilon\eta_2+\frac{3}{4}\epsilon\eta_3-\frac{1}{8}\eta\eta_2-\frac{1}{12}\eta_2^2-\frac{1}{4}\eta_2\eta_3-\frac{1}{24}\eta\eta_3-\frac{1}{12}\eta_3^2-\frac{1}{12}\eta_3\eta_4
		  \right] \, .
		  \label{eq:d4}%\\
		%&\approx \frac{\eta\eta_2(-2\eta_2^2 +\eta(\eta+3)-64\epsilon^2+(35\eta+48)\epsilon+3\eta_2(-\eta-2\eta_3+6\epsilon-2)-\eta_3(\eta+2\eta_3+2\eta_4-18\epsilon+6))}{24\epsilon}\nonumber\\
		%	&\qquad\qquad+\frac{1}{4}(3\eta^2+16\epsilon^2-24\eta\epsilon)\, . 
\end{align}
In the above expressions, equations (\ref{eq:Gamma1})-(\ref{eq:d4}), all slow-roll parameters are evaluated at the time when the mode $k_*$ exits the Hubble horizon and the background field has value $\varphi_*$, so they should actually read $(\epsilon_*, \eta_*, (\eta_2)_*, (\eta_3)_*, (\eta_4)_*)$ but we have omitted the $*$ for the sake of simplifying the notation. Moreover, assuming that the slow-roll approximation holds, it is safe to assume that the last slow-roll parameter $ (\eta_4)_*$ is negligible, and so we are left with a 5-parameters model of the local $\mathcal{P}_\zeta(k)$ around $k_*$,
\be
	\vec{\eta}\equiv\left(V(\varphi_*),\, \epsilon_*, \,\eta_*, \, (\eta_2)_*, \,(\eta_3)_*\right)\,.
\ee
We also expect a hierarchy between the different parameters, i.e., $1> (\epsilon_*, \,\eta_*)> (\eta_2)_*>(\eta_3)_*$.

\bigskip

We are now left with the problem of converting measurements at different $k$ for $|\zeta_k|$ into corresponding measurements at different $\Delta\varphi$. To do so, we use the identity $k=aH$, and the identity
\be
	\ln a \equiv \int  \mathrm{d}t \, H(t) \, = \, \int  \mathrm{d}\varphi \frac{H}{\dot\varphi}\, =\, \int \mathrm{d}\varphi \frac{1}{\sqrt{2\epsilon}\Mp}\, .
\ee
This leads to
\bea
	k&=&\exp\left\{ \int \frac{\mathrm{d}\varphi}{\Mp}(2\epsilon)^{-1/2}  \right\}\frac{1}{\Mp}\sqrt{\frac{V}{3-\epsilon}}\, ,\\
		&=&  \frac{1}{\Mp}\exp\left\{ \int \frac{\mathrm{d}\varphi}{\sqrt{2}\Mp}\frac{1}{\sqrt{\epsilon_*+\Gamma_1\frac{\Delta\varphi}{\Mp}+\Gamma_2\frac{\Delta\varphi^2}{2\Mp^2} +\Gamma_3\frac{\Delta\varphi^3}{3!\Mp^3}+\Gamma_4\frac{\Delta\varphi^4}{4!\Mp^4} }}  \right\}
		\nonumber
		\\
		\label{eq:kDeltavarphimapping}
		&&\qquad\qquad\qquad \qquad \qquad\times \sqrt{\frac{V(\varphi_*)\left[1+d_1\frac{\Delta\varphi}{\Mp}+\frac{1}{2}d_2\frac{\Delta\varphi^2}{\Mp^2}+\frac{1}{3!}d_3\frac{\Delta\varphi^3}{\Mp^3} +\frac{1}{4!}d_4\frac{\Delta\varphi^4}{\Mp^4}\right]}{3-\epsilon_*+\Gamma_1\Delta\frac{\varphi}{\Mp}+\Gamma_2\frac{\Delta\varphi^2}{2\Mp^2} +\Gamma_3\frac{\Delta\varphi^3}{3!\Mp^3}+\Gamma_4\frac{\Delta\varphi^4}{4!\Mp^4} }}  \,  .
\eea
 
 The solution to the integral can be given in terms of elliptical functions. For every $k$ this has to be inverted and a value of $\Delta \varphi$ has to be found, but this cannot be done analytically. At this point I think that for every choice of $\vec{\eta}$, the solution will have t be found numerically.
 
 \section{Slow roll parameters inference}
 
 We want to infer the probability of a given vector of slow-roll parameters $\vec{\eta}$ given the observations:
 \be
 	\mathcal{P}(\vec{\eta}| f_n, a_{lm})=\frac{\mathcal{P}( f_n, \, a_{lm}|\vec{\eta})\mathcal{P}(\vec{\eta})}{\mathcal{P}( f_n,\, a_{lm})}=\frac{\mathcal{P}( a_{lm}| f_n,\,\vec{\eta}) \mathcal{P}( f_n|\vec{\eta})\mathcal{P}(\vec{\eta})}{\mathcal{P}( f_n, \, a_{lm})} = \frac{\mathcal{P}( a_{lm}| f_n) \mathcal{P}( f_n|\vec{\eta})\mathcal{P}(\vec{\eta})}{\mathcal{P}( f_n, \, a_{lm})}
 \ee
 %But $\vec{\eta}$ and $f_n$ are conditionally independent given $a_{l,m}$, so $\mathcal{P}(\vec{\eta}| f_n, a_{lm})=\mathcal{P}(\vec{\eta}| a_{lm})$, and, similarly, 
 and in the last equality we used that $a_{l,m}$ and $\vec{\eta}$ are conditionally independent given $f_n$, therefore $\mathcal{P}( a_{lm}| f_n, \,\vec{\eta})= \mathcal{P}( a_{lm}| f_n)$. 
 
 Marginalizing over the potential map (the $f_n$s), we obtain:
%  \be
% 	\mathcal{P}(\vec{\eta}| a_{lm} )=\frac{\mathcal{P}(\vec{\eta})}{\mathcal{P}( f_n, \, a_{lm})}\int \mathrm{d}f_n\mathcal{P}( a_{lm}| f_n) \mathcal{P}( f_n|\vec{\eta})
% \ee
 \be
 	\mathcal{P}(\vec{\eta}| a_{lm} )\propto\mathcal{P}(\vec{\eta})\int \mathrm{d}f_n\mathcal{P}( a_{lm}| f_n) \mathcal{P}( f_n|\vec{\eta})
 \ee

We have:
\be
	\mathcal{P}( a_{lm}| f_n) \propto \exp\left[ -\frac{1}{2}\left(a_{lm}-\mathbf{R} f_n\right)^{\mathrm{T}}C^{-1}_a\left(a_{lm}-\mathbf{R} f_n\right)\right]\, ;
\ee
and assuming the CMB fluctuations are Gaussian, they will be described by a Gaussian distribution with variance given by the inflationary power spectrum:
\be
	\mathcal{P}( f_n|\vec{\eta})=\exp\left[ f_n^{\mathrm{T}}C_f^{-1}f_n \right]\,
\ee
with $C_f^{-1}$ a diagonal matrix given by:
\be
	C_f^{-1}  =\left(  \begin{array}{ccccc} 1/\sigma_{n_1}^1&0&0&... &0 \\ 0& 1/\sigma_{n_2}^2 &0& &... \\ ... & & & &\\ 0&...&  & &1/\sigma_{n_i}^2\end{array} \right)\, ,
\ee
where the $\sigma_{n_i}$ only depend on the norm of the $\vec{n}_i$ vector to which they correspond (or equivalently on the norm or the corresponding Fourier mode $|\vec{k}|$). More specifically, they are given in terms of the power spectrum by:
\bea
	\sigma_k^2&=&\mathcal{P}_\zeta(k)\nonumber\\
	&=&\frac{1}{4k^{3}}\frac{V(\varphi_*)\left[1+d_1\frac{\Delta\varphi}{\Mp}+\frac{1}{2}d_2\frac{\Delta\varphi^2}{\Mp^2}+\frac{1}{3!}d_3\frac{\Delta\varphi^3}{\Mp^3} +\frac{1}{4!}d_4\frac{\Delta\varphi^4}{\Mp^4}\right]}{\Mp^2\left( \epsilon_*+\Gamma_1\frac{\Delta\varphi}{\Mp}+\frac{\Gamma_2}{2}\frac{\Delta\varphi^2}{\Mp^2}+\frac{\Gamma_3}{3!}\frac{\Delta\varphi^3}{\Mp^3}+\frac{\Gamma_4}{4!}\frac{\Delta\varphi^4}{\Mp^4})\right)\left(3-\epsilon_*+\Gamma_1\frac{\Delta\varphi}{\Mp}+\frac{\Gamma_2}{2}\frac{\Delta\varphi^2}{\Mp^2}+\frac{\Gamma_3}{3!}\frac{\Delta\varphi^3}{\Mp^3}+\frac{\Gamma_4}{4!}\frac{\Delta\varphi^4}{\Mp^4}\right)}\, .
\eea
Here, the mapping between $k$ and $\Delta \varphi$ has to be done using equation (\ref{eq:kDeltavarphimapping}). We therefore obtain that the $f_n$ covariance matrix only depends on the $\vec{\eta}$ vector. To completely define the problem, we are left with the task of specifying a prior on $\vec{\eta}$. This can be done using the following constraints:
\be
	\left\{\begin{array}{ccl} V(\varphi_*)&\rightarrow &\mathrm{fixed~by~}H_*\, \varepsilon\, \left[10^{11.5}\, \mathrm{GeV},\,  10^{15}\, \mathrm{GeV}\right] \\
	\epsilon_* &\varepsilon& \left[10^{-6},\, 1 \right[\\
	\eta_*  &\varepsilon& \left]-1,\, 1 \right[\\
	(\eta_2)_*  &\varepsilon& \left[-1,\, 1 \right]\\
	|(\eta_3)_*|  &<& |(\eta_2)_*|
	\end{array} \right.
\ee
\bigskip

Performing the integral over $\mathrm{d}f_n$, we obtain (see handwritten notes for details):
\be
	\mathcal{P}(\vec{\eta}| a_{lm} )\propto\frac{(2\pi)^{(n_i/2)}}{\mathrm{Det}\,\mathbf{A}}\exp\left\{\frac{1}{2}\mathbf{B}^{\mathrm{T}}\mathbf{A}^{-1}\mathbf{B}-\frac{1}{2}a^{\mathrm{T}}_{lm}C_a^{-1}a_{lm} \right\}\mathcal{P}(\vec{\eta})\, ,
\ee  
where the $\mathbf{A}$ and $\mathbf{B}$ matrices are given by:
\bea
	\mathbf{A}~~&=&\mathbf{R}^{\mathrm{T}}C^{-1}_a \mathbf{R}+C_f^{-1}\\
	 \mathbf{B}^{\mathrm{T}}&=&a_{lm}^{\mathrm{T}}C_a^{-1}\mathbf{R}
\eea





\end{document}
