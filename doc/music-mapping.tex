%%%%%%%%%%%%%%%%%%%%%%%%%%%%%%%%%%%%%%%%%%%%%%%%%%%%%%%%%%%%%%%%%%%%%%

\documentclass[useAMS,usenatbib,a4paper]{mn2e}

\voffset=-0.6in

% Packages:
\input psfig.sty
\usepackage{xspace}
\usepackage{graphicx}
\usepackage{amssymb}
\usepackage{amsmath}

% Macros:
% Addresses
\def\kipac{Kavli Institute for Particle Astrophysics and Cosmology, Stanford University, 452 Lomita Mall, Stanford, CA 94035, USA}
\def\rdbemail{\texttt{rdb3@stanford.edu}}
\def\lplemail{\texttt{levasseur@stanford.edu???}}

% Making life easier
\newcommand{\be}{\begin{equation}}
\newcommand{\ee}{\end{equation}}
\newcommand{\bs}{\begin{split}}
\newcommand{\bea}{\begin{eqnarray}}
\newcommand{\eea}{\end{eqnarray}}

% Useful symbols
\newcommand{\om}{\Omega_m}
\newcommand{\ha}{\frac{1}{2}}
\newcommand{\ahub}{\frac{\dot{a}}{a}}
\newcommand{\ode}{\Omega_{de}}
\newcommand{\Oe}{\Omega_{e}}
\newcommand{\lcdm}{$\Lambda$CDM}
\newcommand{\neff}{N_{\rm eff}}
\newcommand{\hfid}{H^2_{\rm fid}}
\newcommand{\dl}{\delta}
\newcommand{\sumu}{\Sigma m_\nu}
\newcommand{\mnu}{\Sigma m_\nu}
\newcommand{\mpci}{\,{\rm Mpc}^{-1}}

\newcommand{\dataext}{\data_{\rm ext}}
\newcommand{\transferext}{\mathbf{R}_{\rm ext}}
\newcommand{\smat}{\mathbf{S}}
\newcommand{\cmat}{\mathbf{C}^{\epsilon}}
\newcommand{\cmatext}{\mathbf{C}^{\epsilon_{\rm ext}}}
\newcommand{\noisemat}{\mathbf{N}^{\epsilon}}
\newcommand{\noisematext}{\mathbf{N}^{\epsilon_{\rm ext}}}
\newcommand{\noisematinv}{\left(\noisemat\right)^{-1}}
\newcommand{\noisemattransfer}{\tilde{\mathbf{N}}^{\epsilon}}
\newcommand{\noisemattransferext}{\tilde{\mathbf{N}}^{\epsilon_{\rm ext}}}
\newcommand{\noisemattransferinv}{\left(\tilde{\mathbf{N}}^{\epsilon}\right)^{-1}}
\newcommand{\noisemattransferextinv}{\left(\tilde{\mathbf{N}}^{\epsilon_{\rm ext}}\right)^{-1}}

% Macros
\newcommand{\half}{\frac{1}{2}}
\newcommand{\rhocrit}{\rho_{\rm crit}}
\newcommand{\rvir}{r_{\rm vir}}
\newcommand{\mvir}{m_{\rm vir}}
\newcommand{\kv}{\mathbf{k}}
\newcommand{\xv}{\mathbf{x}}
\newcommand{\mv}{\mathbf{m}}
\newcommand{\muv}{\bm \mu}
\newcommand{\hmsun}{h^{-1}M_{\odot}}
\newcommand{\hmpc}{h^{-1}Mpc}
\newcommand{\hgpc}{h^{-1}{\rm Gpc}}

%%% "Data" - i.e., the observed CMB temperature map
\newcommand{\data}{\mathbf{d}}
%%% Parameters: gravitational potential
\newcommand{\Pot}{\Phi}
% \newcommand{\Potvector}{\boldsymbol{\Phi}}
\newcommand{\Potvector}{\Pot}
%%% "Signal" - i.e., zero-noise CMB temperature
\newcommand{\signal}{\mathbf{s}}
%%% "noise" - i.e., the pixel noise realization
\newcommand{\noise}{\mathbf{n}}
%%% Transfer function relating the 3D gravitational potential to the
%%% 2D CMB temperature (or other) map
\newcommand{\transfer}{\mathsf{R}}
%%% Signal and noise covariance matrices
\newcommand{\Smat}{\mathsf{S}}
\newcommand{\Nmat}{\mathsf{N}}
\newcommand{\Psimat}{\mathsf{\Psi}}
\newcommand{\Sigmamat}{\mathsf{\Sigma}}
\newcommand{\Cmat}{\mathsf{C}}
%%% Gravitational potential represented as a vector of "voxels" or similar
\newcommand{\gravpot}{\bm \Psi}
%%% Normal (Gaussian) distribution
\newcommand{\normdist}{\mathcal{N}}
%%% Probability theory
\newcommand{\pr}{{\rm Pr}}

% Alternative nomenclature:
\def\Response{\mathsf{R}_{yn}}
\def\ResponseMatrix{\transfer}

% Journals:
\newcommand{\apj}{ApJ}
\newcommand{\apjl}{ApJL}
\newcommand{\apjs}{ApJS}
\newcommand{\mnras}{MNRAS}
\newcommand{\apss}{Ap \& SS}
\newcommand{\aap}{A\&A}
\newcommand{\aj}{AJ}
\newcommand{\prd}{Phys. Rev. D}
\newcommand{\nat}{Nature}
\newcommand{\araa}{ARA\&A}
\newcommand{\jgr}{J. Geophys. Res.}
\newcommand{\pasp}{PASP}

% Miscellany
\newcommand{\etal}{et~al.~}
\newcommand{\eg}{{\it e.g.\ }}
\newcommand{\ie}{{\it i.e.\ }}
\newcommand{\etc}{{\it etc.\ }}



% Lauerence's stuff:

\usepackage{color}% use if color is used in text
\newcommand{\tc}{\textcolor{red}}

\newcommand{\rc}{\nonumber\\}

\newcommand{\rhophi}{\rho_{\rm DDE}}
\newcommand{\de}{\partial}
\newbox\pippobox

\def\bx{\bm{x}}
\def\bk{\bm{k}}
\def\tpk{{\tilde{\phi}_k}}

%\newcommand{\eqn}[1] {Eq.~(\ref{#1})}
%\newcommand{\fig}[1] {Fig.~(\ref{#1})}
\newcommand {\lla} {\ {\raise-.5ex\hbox{$\buildrel<\over\sim$}}\ }
\renewcommand{\(}{\left(}
\renewcommand{\)}{\right)}
\renewcommand{\[}{\left[}
\renewcommand{\]}{\right]}

\def\iPP{\color{red}}
\def\bl{\color{green}}

\usepackage[countmax]{subfloat}
\usepackage[T1]{fontenc}
\usepackage[latin1]{inputenc}
\usepackage{graphicx}
\usepackage[english]{babel}
\usepackage{amsmath}
\usepackage{amssymb}
\usepackage{amsfonts}

\def\pp{{\, \mid \hskip -1.5mm =}}
\def\cL{{\cal L}}
\def\tr{{\rm tr}\, }
\def\nn{\nonumber \\}
\def\e{\mathrm{e}}

\def\be{\begin{equation}}
\def\ee{\end{equation}}
\def\bea{\begin{eqnarray}}
\def\eea{\end{eqnarray}}
\newcommand{\ex}{\mathrm{e}}
\newcommand{\dd}{\mathrm{d}}
\newcommand{\gsim}{\gtrsim}
\newcommand{\lsim}{\lesssim}
\def\spose#1{\hbox to 0pt{#1\hss}}
\def\Vec#1{\mbox{\boldmath$#1$\unboldmath}}
\def\lta{\mathrel{\spose{\lower 3pt\hbox{$\mathchar"218$}}
     \raise 2.0pt\hbox{$\mathchar"13C$}}}
\def\gta{\mathrel{\spose{\lower 3pt\hbox{$\mathchar"218$}}
     \raise 2.0pt\hbox{$\mathchar"13E$}}}
\newcommand{\trace}{\mathrm{Tr}}
\def\setH{\mathbb{H}}
\def\setR{\mathbb{R}}
\def\setC{\mathbb{C}}
\def\setG{\mathbb{G}}
\def\setO{\mathbb{O}}
\def\setN{\mathbb{N}}
\def\setZ{\mathbb{Z}}
\def\setUni{\mathbb{1}}

% Cosmology
\newcommand{\Hu}{\mathcal{H}}
\newcommand{\Ka}{\mathcal{K}}
\newcommand{\Zed}{\mathcal{Z}}
\newcommand{\cs}{c_{_\mathrm{S}}}
\newcommand{\ns}{n_{_\mathrm{S}}}

% General Physics
\newcommand{\GN}{G_{_\mathrm{N}}}
\newcommand{\mP}{m_{_\mathrm{Pl}}}
\newcommand{\Mp}{M_{_\mathrm{Pl}}}
\newcommand{\lP}{\ell_{_\mathrm{Pl}}}
\newcommand{\Lag}{\mathcal{L}}


%%%%%%%%%%%%%%%%%%%%%%%%%%%%%%%%%%%%%%%%%%%%%%%%%%%%%%%%%%%%%%%%%%%%%%

\title[The 3D Potential of the Universe from CMB Data]
{The Music of the Sphere: Inferring the 3D Gravitational Potential of the Universe on the Largest Scale from Cosmic Microwave Background Observations}

\author[Blandford et al.]{%
    Roger~D.~Blandford,$^{1}$\thanks{\rdbemail}
    Philip~J.~Marshall,$^{1}$
    Laurence Perrault Levasseur$^{1}$
    \medskip\\
    $^1$\kipac
}


%%%%%%%%%%%%%%%%%%%%%%%%%%%%%%%%%%%%%%%%%%%%%%%%%%%%%%%%%%%%%%%%%%%%%%

\begin{document}

\date{to be submitted to arxiv}
\pagerange{\pageref{firstpage}--\pageref{lastpage}}\pubyear{2015}

\maketitle

\label{firstpage}

%%%%%%%%%%%%%%%%%%%%%%%%%%%%%%%%%%%%%%%%%%%%%%%%%%%%%%%%%%%%%%%%%%%%%%

\begin{abstract}

Abstract goes here.

\end{abstract}

% Full list of options at http://www.journals.uchicago.edu/ApJ/instruct.key.html

\begin{keywords}
  cosmology
\end{keywords}

\setcounter{footnote}{1}

%%%%%%%%%%%%%%%%%%%%%%%%%%%%%%%%%%%%%%%%%%%%%%%%%%%%%%%%%%%%%%%%%%%%%%

\section{Overview}

Observations of temperature fluctuations in the CMB measure the 2D
potential $\Phi$ (considered as a linear perturbation to the
Robertson-Walker metric tensor under the Newtonian gauge) on the
sphere of last scattering (where the scale factor $a\sim0.0009$ and
the time is $t\sim380$~kyr). The measurement is quite direct on large
angular scale ($\ell\lesssim30$) in the ``Sachs-Wolfe'' regime; it is
indirect on small angular scales where velocity and density
perturbations are more important, which are linearly related to the
potential perturbation. It is also possible, at least in principle, to
learn about the first and second radial derivatives of this potential
through studying the polarization. This 2D potential is derived from a
3D potential which fills the sphere and extends beyond our horizon.
This potential is derived from one specific realization of an initial
Fourier spectrum of inferred type and statistical properties.  This
paper reports on an investigation of what can be learned, at least in
principle, about the 3D potential at the time of recombination from
the 2D information. Furthermore, any set of Fourier components can be
evolved assuming the now standard ``Flat $\Lambda$ CDM'' cosmology and
connected to the potential information derivable from large surveys
conducted out to modest redshift. What is being discussed here is an
exercise in basic cartography and not in measuring the physical
behavior of the universe at either early or late times. As with many
such exercises the goal is to understand the approximate arrangement
of structure within the observable universe on the largest linear
scales. We shall be more concerned with the topological organization
of this structure than with precise measurement. However, success in
this endeavor ought to improve the investigation of physics questions
through contributing constraining priors to Bayesian inference
studies.

In Sec. 2, we discuss the basic assumptions that we will make in our
idealized versions of the problem. This is followed in Sec. 3 by a
description o fate Fourier representation of the 3D potential and its
relationship to the spherical harmonic decomposition on the last
scattering sphere. The nesting of equipotential contours on the last
scattering photosphere is analyzed in Sec. 4 and this is generalized
to surfaces in 3D in the following section.  The relationship between
the 2D and 3D equipotentials is discussed in Sec.6. In Sec. 7., we
outline a Bayesian approach to calculating the likeliest form of the
3D potential, paying special attention to the criteria which dictate
the effective resolution of the reconstruction and how this may be
improved by adding interior data. Our conclusions are collected in
Sec. 8. A future publication will apply this approach to actual data.

%%%%%%%%%%%%%%%%%%%%%%%%%%%%%%%%%%%%%%%%%%%%%%%%%%%%%%%%%%%%%%%%%%%%%%

\section{Basic Assumptions}

We  work with an idealized problem which we specify as follows:

\begin{itemize}
\item Represent the big bang as a sphere of comoving radius $\sim14.23$~Gpc (adopting a Hubble constant of 68 km s$^{-1}$ Mpc$^{-1}$) and flat $\Lambda$CDM .  The last scattering surface -- the \emph{cosmic photosphere} --  is a sphere with radius $\sim0.29$~Gpc smaller. Recombination occurs over a short but finite interval of time just prior to last scattering. The radius of the cosmic photosphere, 13.94~Gpc, is our unit of length.
\item Ignore the expansion of the universe and just consider the potential as a set of linear normal modes at the time of last scattering. These can be evolved backward and forward in time with confidence. In practice, the potential changes little after recombination although shorter-wavelength modes eventually become nonlinear. (This, too, can be accommodated in principle.)
\item Consider only scalar modes, setting the tensor contribution to zero. If, one day, a significant tensor component is confidently measured, minor modifications to what follows can be included.
\item Hypothesize that the amplitude of each of these modes is drawn from a Gaussian distribution with zero mean, random phase and initial variance roughly proportional to $k^{-3}$ as is consistent with the observations.
\item Ignore modes with wavelength longer than the side of the box, {L}. Their contributions can be approximately incorporated into the lowest Fourier components in a given box as long as we only care about observations within our horizon. This procedure will improve as we enlarge the box. However if the box is too large the spacing of modes in k-space $\Delta k=2\pi/L$ will be too fine and the amplitudes of neighboring modes will be poorly distinguished. A choice $L\sim4$ turns out to be a good compromise. Truncate the spectrum with a sphere of radius $n_{\rm max}\Delta k$; shorter wavelength modes contribute to the error and a Gaussian window function may be preferable.
\end{itemize}

%%%%%%%%%%%%%%%%%%%%%%%%%%%%%%%%%%%%%%%%%%%%%%%%%%%%%%%%%%%%%%%%%%%%%%

\section{Fourier Modes}
We represent the potential $\Phi$ within the box in polar coordinates
as a Fourier series with wave vectors ${\bf k}=\Delta
k\{n_1,n_2,n_3\}$, with $n_1,n_2,n_3$ integers and $\Delta k=2\pi/L$.
As $\Phi$ is real, we only need to assign one real number to each
mode. We approximate Eq.~(1) by a finite sum restricted to
$(n_1^2+n_2^2+n_3^2)^{1/2}\le n_{\rm max}$ and we label the
coefficients $f_{\bf k}$ by the index $n$ running from $1$ to
$N\sim4\pi n_{\rm max}^3/3$. ($N=6,32,122,
256,514,924,1418,2108,3070,4168$ for $n_{\rm max}=1$ through $10$.) It
is simplest to restrict ${\bf k}$ to a hemisphere and to write:
\begin{equation}
\Phi[{\bf x}(x,\theta,\phi)]=\sum_{n=1}^{N/2}[f_n\cos({\bf k}_n\cdot{\bf x})+f_{N+1-n}\sin({\bf k}_n\cdot{\bf x})]
\end{equation}

\begin{figure*}
\centering
\includegraphics[width=0.9\linewidth]{fig1.jpg}
\caption{Aitoff projections of the potential on the cosmic photosphere $\Phi(\theta,\phi;\ell)$ at $x=1$ for different angular resolutions, parametrized by $\ell$. The contour levels are $0\pm1$. A fixed set of random Fourier components truncated with $n_{max}=3$ is used.}
\end{figure*}

Our main goal is to relate surface information on the photosphere to
the underlying $f_n$.\footnote{This is sometimes called
\emph{holography}, though it is not the original meaning of the word.}
The approach that we will follow is constructive. $\Phi$ can be
expanded exactly as an infinite sum of Legendre polynomials which we
truncate at a finite value of $\ell$, starting with $\ell=1$.

\begin{equation}
\Phi({\bf x;\ell})=\sum_{\ell'=0} ^\ell(2\ell'+1)\sum_{n=1}^Nj_{\ell'}(k_nx)P_{\ell'}({\hat{\bf k}}_n\cdot{\hat{\bf x}}){\cal S}(n,\ell')f_n,
\end{equation}
where ${\bf k}_n={\bf k}_{N+1-n}$ and ${\cal S}(n,\ell)=[\cos(\ell\pi/2),\sin\ell\pi/2)]$, for $[1\le n\le N/2,N/2<n\le N]$. As $\ell$ is increased the effective resolution of $\Phi$, in radius and angle, improves. It is convenient to treat $\ell$ as a continuous variable by the device of summing up to $\lfloor\ell\rfloor$  and then adding the next term in the sum multiplied by $(\ell-\lfloor\ell\rfloor)$. To describe the potential generated by modes with a given value of $n_{\rm max}$, we need spherical harmonics up to $\ell\sim3n_{\rm max}$ and {\it vice versa} (Fig.~1). An immediate implication is that an accurate $\Phi$ map on the sphere, degraded to resolution $\ell$ provides $L=(\ell-1)(\ell+3)$ real numbers which can be used to solve for $N$ real Fourier coefficients suggesting that there is enough information in a $\ell=9$ map to solve for $\sim100$ Fourier components up to $n_{\rm max}\sim3$ independent of the priors on their actual values. When we include the priors, we can proceed to finer scale. Also, when we reduce the radius of the sphere, the value of $n_{\rm max}$  probed by a given $\ell$ scales $\propto x^{-1}$ (Fig. 2).

\begin{figure*}
\centering
\includegraphics[width=0.9\linewidth]{fig2.jpg}
\caption{Aitoff projections of the potential on a sphere with $x=0.5$ with the same Fourier series as in Fig.~1. The expansion up to $\ell=4$ is a very good approximation to the full potential.}
\end{figure*}


%%%%%%%%%%%%%%%%%%%%%%%%%%%%%%%%%%%%%%%%%%%%%%%%%%%%%%%%%%%%%%%%%%%%%%

\section{Relating Surface and Interior Equipotentials}

Now let us outline how to extract information about the interior
potential from the surface potential. We suppose that we are in the
Sachs-Wolfe region of the spectrum where the surface potential
$\Phi(1,\theta,\phi)=3\delta_T$, where $\delta_T$ is the relative CMB
fluctuation. We set aside for the moment the velocity and density
contributions to the observed fluctuation and the additional
information (roughly twice as much) that can be garnered from adding
polarization maps. We align the 3 axis (Eq.~(1)) with $\theta=0$ and
the 1 axis with $\phi=0$.

Our first task is to describe the inferred potential on the sky. We
suppose this is approximated by a finite sum over spherical harmonics:

\begin{equation}
\Phi(1,\theta,\phi;\ell_{\rm max})=\sum_{y=1}^{(\ell_{\rm max}+1)^2}a_yY_y(\theta,\phi),
\end{equation}
where $Y_y=\{Y_{0,0},Y_{1,0},2^{1/2}\Re[Y_{1,1}],2^{1/2}\Im[Y_{1,1}],Y_{2,0},\dots,2^{1/2}\Im[Y_{\ell_{\rm max},\ell_{\rm max}}]\}$. Note that there are $2\ell+1$ independent, real, basis function in each $\ell$-shell. Note also that $\int d\Omega Y_yY_{y'}=\delta_{yy'}$.

Next, we describe the potential using Eq.~(2) and the identity
\begin{equation}
P_{\ell}[\hat{\bf k}(\theta',\phi')\cdot\hat{\bf x}(\theta,\phi)]=\frac{4\pi}{2\ell+1}\sum_{y=\ell^2+1}^{(\ell+1)^2}Y_y(\theta',\phi')Y_y(\theta,\phi)
\end{equation}
The spherical harmonic coefficients are then given by the linear relation
\begin{equation}
a_y=\Response f_n,
\end{equation}
where we adopt the summation convention and
\begin{equation}
\Response =4\pi i^\ell Y_y^\ast(\theta',\phi')j_\ell(k),
\end{equation}
with $\cos\theta'=n_3/n,\tan\phi'=n_2/n_1$. The elements of the
transformation matrix $\Response$ have been tabulated up to $n_{\rm
max}=6$, $l=10$ which should be sufficient for our purpose.

We estimate the importance of spherical harmonics with order $\ell$ to
the signal at a given given range of $k$ by evaluating the sum:
\begin{equation}
P(\ell;n_{\rm max})=\sum_{n=N(n_{\rm max}-1)+1}^{N(n_{\rm max})}\sum_{y=\ell^2-3}^{(\ell-1)(\ell+2)}|\Response|^2
\end{equation}
\begin{figure}
\centering
\includegraphics[width=0.9\linewidth]{fig6.jpg}
\caption{Importance of spherical harmonics of order $\ell$ for shells
in k-space with inner radius $n_{\rm max}-1$ and outer radius
$n_{max}$.}
\end{figure}


%%%%%%%%%%%%%%%%%%%%%%%%%%%%%%%%%%%%%%%%%%%%%%%%%%%%%%%%%%%%%%%%%%%%%%

\section{Inferring the 3D Potential}

Our task is to take a given vector of measured spherical harmonic
coefficients $\mathbf{a}$, with covariance matrix $C$,  and
infer a vector of the Fourier coefficients of our model potential,
$\mathbf{f}$,  under the prior assumption that the  components $f_n$
are independent and drawn from a Gaussian distribution with variance
$\sigma_n^2 = (n_1^2+n_2^2+n_3^2)^{-3/2} / \alpha$. We seek the
posterior PDF ${\cal P} = \Pr(\mathbf{f}|\mathbf{a},\alpha)$, where
\begin{equation}
% -\ln{\cal P} = \frac{(a_y^2-\Response f_n)^2}{2\sigma_y^2}+\frac{f_n^2}{2\sigma_n^2} + \text{const.}
-\ln{\cal P} = (\mathbf{a}-\ResponseMatrix\mathbf{f})^{\rm T} C^{-1} (\mathbf{a}-\ResponseMatrix\mathbf{f})
             + \mathbf{f}^{\rm T} S^{-1} \mathbf{f} + \text{const.}
\end{equation}
and the matrix $S$ is diagonal, with
elements~$S_{nn} = \sigma_n^2$.
Differentiating this expression leads to a set of linear equations
% \begin{equation}
% \left(\frac{\Response m_{yn'}}{\sigma_y^2}+\frac{\delta_{nn'}}{\sigma_n^2}\right)f_{n'}=\frac{a_y\Response}{\sigma_y^2},
% \end{equation}
which can be solved for the maximum posterior 3D potential
Fourier coefficients $\mathbf{f}_{\rm MP}$, given a choice of power spectrum
normalisation~$\alpha$. We follow \citep{SuyuEtal2006} and infer the
most probable normalisation~$\alpha_{\rm MP}$ using the Bayesian Evidence,
which leads to the approximate condition
\begin{equation}
    (\mathbf{a}-\ResponseMatrix\mathbf{f}_{\rm MP})^{\rm T} C^{-1} (\mathbf{a}-\ResponseMatrix\mathbf{f}_{\rm MP})
                 + \mathbf{f}_{\rm MP}^{\rm T} S^{-1}(\alpha) \mathbf{f}_{\rm MP} \approx N/2
\end{equation}
where $N$ is the number of spherical harmonic coefficients used. An
iterative scheme was used to optimize $\alpha$ in this way.

We note that the covariance matrix of the measured spherical harmonic
coefficients should not include any cosmic variance terms, because we
are interested in inferring the 3D potential in our one observable
universe. To estimate this object, we took 100 posterior sample
``Commander-Ruler'' temperature maps, decomposed each of them into
spherical harmonics,\footnote{We use the {\sc healpy} code provided at
\texttt{https://healpy.readthedocs.org}.} and then calculated the
sample variance and sample covariance of the coefficients.


%%%%%%%%%%%%%%%%%%%%%%%%%%%%%%%%%%%%%%%%%%%%%%%%%%%%%%%%%%%%%%%%%%%%%%

\section{Discussion}

%%%%%%%%%%%%%%%%%%%%%%%%%%%%%%%%%%%%%%%%%%%%%%%%%%%%%%%%%%%%%%%%%%%%%%

\bibliographystyle{apj}
\bibliography{references}

%%%%%%%%%%%%%%%%%%%%%%%%%%%%%%%%%%%%%%%%%%%%%%%%%%%%%%%%%%%%%%%%%%%%%%

\end{document}

%%%%%%%%%%%%%%%%%%%%%%%%%%%%%%%%%%%%%%%%%%%%%%%%%%%%%%%%%%%%%%%%%%%%%%
%
%
% \begin{eqnarray}
% \Phi(x,\theta,\phi)&=&8\pi\sum_{\ell=0,2,\dots}^\infty(-1)^{\ell/2}\sum_{\bf k}^{\cal H}\Re[{\tilde\Phi}_{\bf k}]j_\ell(kx)\sum_{m=-\ell}^\ell Y_{\ell m}^\ast(\theta',0)Y_{\ell m}(\theta,0)\cos[m(\phi-\phi')]\cr
% &+&8\pi\sum_{\ell=1,3,\dots}^\infty(-1)^{(\ell+1)/2}\sum_{\bf k}^{\cal H}\Im[{\tilde\Phi}_{\bf k}]j_\ell(kx)\sum_{m=-\ell}^\ell Y_{\ell m}^\ast(\theta',0)Y_{\ell m}(\theta,0)\cos[m(\phi-\phi')].
% \end{eqnarray}
%
% where ${\bf k}=\pi[n_1,n_2,n_3]=k[\sin\theta'\cos\phi',\sin\theta'\sin\phi',\cos\theta']$, with $n_1,n_2,n_3$ integers and ${\bf x}=x[\sin\theta\cos\phi,\sin\theta\sin\phi,\cos\theta]$.  As $\Phi$ is real, we know that ${\tilde\Phi}_{\bf k}(-{\bf k})={\tilde\Phi}^\ast_{\bf k}({\bf k})$ and so we only need define the Fourier components over a hemisphere $\cal H$ in ${\bf k}$ space. We choose ${\cal H}=\{n_1>0\}\cup\{n_1=0,n_2>0\}\cup\{n_1=n_2=0,n_3>0\}$.
%
%
% \begin{eqnarray}
% \Phi(\bf x)&=&2\sum_{\bf k}^{\cal H}{\tilde\Phi}_{\bf k}\sum_{\ell=0}^\infty (2\ell+1)i^\ell P_\ell({\hat{\bf k}}\cdot{\hat{\bf x}})j_\ell(kx)\cr
% &=&8\pi\sum_{\bf k}^{\cal H}{\tilde\Phi}_{\bf k}\sum_{\ell=0}^\infty i^\ell j_\ell(kx)\sum_{m=-\ell}^\ell Y_{\ell m}^\ast(\theta',\phi')Y_{\ell m}(\theta,\phi)
% \end{eqnarray}
%
%  Let us start with the monopolar ($\ell=m=0$) term in Eq.~(3). For the moment, we just explore the partial potential associated with this single term, $\Phi^{00}({\bf x})=\sum{\tilde\Phi}^{00}_{\bf k}e^{i{\bf k}\cdot{\bf x}}$. We can expand this in terms of the original Fourier coefficients as ${\tilde\Phi}^{00}_{\bf k}=M^{00}_{\bf kk'}{\tilde\Phi}_{\bf k'}$, where $\bf k$, $\bf k'$ are shorthand for all the indices that specify individual spatial modes (summed when repeated) and we allow the matrix $M^{00}$ to pick out the real part of $\tilde\Phi_{\bf k}$. In this case,
% \begin{equation}
% M^{00}_{\bf kk'}=\frac14\int d{\bf x}{\rm sinc}kx\cos{\bf k'}\cdot{\bf x},
% \end{equation}
% which is best evaluated numerically. We proceed in this manner to evaluate all the necessary matrix elements $M^{\ell m}_{\bf kk'}$. (Of course there is an infinity of function spaces we could have chosen.)
%
% The strategy, then, is to start with a set of Fourier coefficients, $\tilde\Phi_{\bf k}$ which defines $\Phi({\bf x})$ within the box to sufficient resolution. We then use Eq.~(3) to define partial potentials associated with each spherical harmonic and sum them over $m$ to get a unique partial potential associated with each value of $\ell$. We then sum over all $\ell\le\ell_{\rm max}$ to give a low resolution version of the full potential which we designate as $\Phi(\ell_{\rm max},{\bf x})$. The benefit of this seemingly cumbersome procedure is that we do not have to include large values of $\bf k'$ for it to resemble an appropriately  smoothed version of the full potential and it accurately recovers the 2D potential as defined on the cosmic photosphere of radius $x_\gamma$. Conversely, as we increase $\ell_{\rm max}$, we converge quickly to the true potential at lower spatial resolution (Fig. ~1). An important point for what follows is that we can treat $\ell_{\rm max}$ not as an integer but as a continuous variable simply by adding a constant fraction in $\{0,1\}$  of all the $m$ components associated with $\ell=\ell_{\rm max}$.
%
% We now repeat this exercise for the equipotential surfaces within the sphere. The structurally stable stationary points in 3D are again highs (the Hessian has three negative eigenvalues), lows (three positive eigenvalues) and saddles (both signs). The separatrices are now surfaces passing through the saddles. Close to the saddles, they take the form of two cones with a common vertex on the saddle.
%
% What we must actually do first is to consider the equipotentials defined within the fundamental box with periodic boundary conditions so that opposite faces are identified. It is easy to see that the minimum number of stationary points in three dimensions is eight (six saddles a high and  a low). However, if we start with a sphere of sufficiently small radius, $x<1$ and $\ell=1$, there will be no stationary points within the sphere. We can then increase $x$ to unity bringing in additional stationary points. Next we increase $\ell$ as we did on the sphere. Provided that we consider the entire cube, each saddle is created with an accompanying $H$ or $L$.
