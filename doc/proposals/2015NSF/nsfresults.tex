\documentclass[psfig,11pt]{article}
\usepackage{graphicx}
\setlength{\textheight}{9in}
\setlength{\topmargin}{0in}
\setlength{\baselineskip}{0.1in}
\setlength{\textwidth}{6.3in} 
\setlength{\oddsidemargin}{0in}  
\def\cf{{\it cf.}}
\def\eg{{\it e.g.}}
\def\ie{{\it i.e.}}
\def\etal{{\it et al.}}
\def\etc{{\it etc}}
\def\ni{\noindent}
\def\go{\mathrel{\raise.3ex\hbox{$>$}\mkern-14mu\lower0.6ex\hbox{$\sim$}}}
\def\lo{\mathrel{\raise.3ex\hbox{$<$}\mkern-14mu\lower0.6ex\hbox{$\sim$}}}
\begin{document}
\section{Results from AST08-07458. PI: R. Blandford}
This proposal was to carry out research on gravitational lensing. Some principal results will be summarized here.
\subsection{Strong Lensing}
\subsubsection{Time delays} 
Blandford and Hilbert, lately joined by Hezaveh, as members of a large collaboration led by former student Suyu and also including KIPAC member Marshall, refined  the measurement of the Hubble constant and other cosmological parameters using the gravitational lens B1608+656. They successfully tested and implemented new formalisms to include the perturbative effects of intervening galaxies and large scale structure upon the time delays.  They have also used the Millennium simulation to devise a procedure for estimating the statistical distribution of the mean convergence along the line of sight, given a catalog of observations of nearby galaxies. This improves the accuracy with which lensing determinations of the Hubble constant can be achieved with imminent observations. Recent research has been extended to RXJ1131-1231, where the effect of extrinsic convergence should be lower than for B1608+656. This has already furnished a seven percent distance solution. More generally, the convergence distributions are also important for galaxy counts or halo models based upon photometry. There has been recent tension between Planck-based determinations of the Hubble constant and lensing on most other approaches. Some progress has been made on clarifying the practical strengths and limitations of lens approaches.
\subsubsection{Galactic structure and Initial Mass Function} 
Barnabe continued to refine the CAULDRON code that he developed for his thesis that combines stellar dynamics with gravitational lensing. He then applied it to observations made under the Sloan Lens ACS Survey (SLACS). The main galaxies of study were early type and it was possible to study their IMF, especially their low mass cut offs and to deduce that there had been little evolution since z~0.35. Barnabe has also been a leader of the Sloan WFC Edge on Late type Lens Survey  (SWELLS). This is a major project to model edge on spirals that act as gravitational lenses.  The goal is to obtain a better understanding of the dynamical structure of spiral galaxies and, especially, to study the stellar distribution and to infer the  initial mass functions associated with the different components of the galaxy � the disk, bulge and the halo. Over several publications it has been concluded that there is no universal IMF and that 
\subsection{ALMA}
Hezaveh and collaborators have demonstrated that ALMA observations of  strong gravitational lenses can be used to provide a gravitational measurement of the power spectrum of dark matter subhalos. Observations will be taken shortly which have the potential to validate the particulate nature of dark matter. Hezaveh has proposed that high redshift galactic nuclei that are strongly lensed may have their gas kinematics well enough resolved to furnish reliable black holes masses.  This technique will soon be practical with ALMA and eventually with GSMTs. A paper has been published.
\subsection{Cosmic Shear}
\subsubsection{Weak lensing surveys}
Hilbert and others used shear correlations from the Deep Lens Survey to derive constraints on the cosmic mean matter density and the amplitude of matter fluctuations. In particular, Hilbert contributed cosmic shear data covariance matrices. The first full analysis has been published and a tomographic extension is in preparation. Hilbert and colleagues developed a new method for detecting shear peaks in weak lensing surveys and studying their abundance and spatial correlation using simulations, They showed how to constrain more effectively cosmological parameters from weak lensing surveys when this analysis was combined with complementary measurements. This approach is currently being applied ot 
 Looking to the future, Hilbert has simulated the cosmological constraints one could obtain from an LSST or a Euclid weak lensing survey by combining shear correlations, aperture mass statistics, and shear peak counts. He demonstrated that competitive constraints on cosmological parameters, including non-Gaussianity would be possible. Such observations might thus provide valuable constraints on models of inflation and the physics of the very early Universe. Schrabback, who has been designated as the deputy lead for Euclid weak lensing shape measurements, has studied the calibration and the influence of color gradients in galaxies on the accuracy
\subsubsection{Improving photometric redshifts}
Schrabback and others have re-analyzed the CFHT Legacy Survey as part of the CFHT Lens Survey. They have paid particular attention to performing more accurate photometry and used the results to improve the determination of photometric redshifts. The results of this exercise are mixed with the most positive outcome being an improved PSF procedure which may lead ultimately to improved redshift determinations.
\subsubsection{Galaxy shapes, intrinsic alignment, and weak lensing}
Hilbert and colleagues used simulations by Springel of cosmic structure to make a more direct assessment of the impact of intrinsic alignment on cosmic shear surveys.  Two papers have been published and a third is in preparation. Hilbert, Schrabback and colleagues then investigated the possible contamination of weak lensing measurements by intrinsic alignments of galaxy shapes. They quantified the shape distribution of various galaxy samples in the COSMOS survey and compared the results to predictions from N-body simulations finding that simple models of galaxy shapes in the literature fail to reproduce the observed shape distribution. Currently, they are considering a broader set of semi-analytic models models to understand this discrepancy. Schrabback and colleagues have made a study of the influence of color gradients on the shapes of faint galaxies used for gravitational lensing investigations of dark energy. They find that these effects can be significant but are correctable and give prescriptions for keeping the errors that they may engender below statistical errors.
\subsection{Galaxy-Galaxy Lensing}
\subsubsection{Light-matter correlation}
The CFHT Legacy Survey-Wide is the most powerful data set for weak lensing measurements currently existing. It comprises 170 square degrees of sky imaged in five bands.  Schrabback is a core member of the team which is conducting a thorough analysis of the complete dataset, which is now corrected for systematic bias and includes all the available photometric redshift information. The primary goals of the survey are to explore the relationship between luminous and dark matter and to place significant constraints on cosmological parameters. He has also completed an analysis of  the flattening of halos. Hilbert et al. used N-body simulations of structure formation along with semi-analytic galaxy-formation models and ray-tracing to show that higher-order galaxy-galaxy lensing correlations can be used to provide new information about galaxy formation. An analysis of galaxy-galaxy lensing in the CFHTLS yielded a number of unexpected results for the relation between matter and galaxies which may be a consequence of systematic error. A paper on lensing 2- and 3-point correlations has been published and another paper is in preparation.
\subsubsection{Evolution of the ratio of stellar to dark mass in galaxies}
Schrabback and colleagues have used data from the COSMOS field to perform a new investigation of how the light and mass from galaxies varies with redshift out to z=1. They are able to measure the �downsizing at modest redshift and discuss some mechanisms involving AGN feedback and disk instabilities which may be responsible.
\subsection{Clusters}
\subsubsection{Weak lensing observations of rich clusters of galaxies}
Schrabback has also been devising new techniques for galaxy shape measurements in weak lensing studies of rich clusters of galaxies. The goal is to increase the number density of background galaxies that can be included. These techniques have been successfully applied mainly to HST-ACS observations. In particular they are being used by von der Linden in the analysis of MACS 0717.5$+$3745, which also includes wide-field imaging from the ground. The pipeline now includes the merging system MACSJ0417.5-1154 as well as several clusters from the HST Multi-Cycle Treasury Program CLASH, and discovered by the South Pole Telescope. Studies of the cosmological evolution of clusters are limited by the small number of high redshift clusters suitable for comboined strong and weak lensing analysis. In order to increase the sample, Schrabback is combining the ROSAT X-ray Survey and the Sloan Digital Sky Survey samples. This involves optical imaging with the WHT and LBT telescopes, and radio (Sunyaev-Zel'dovich) observations with the CARMA (SZA) Array. Chandra and HST observations have also been proposed. Von der Linden has led a large team including Blandford on an ambitious project to carry out a systematic weak lensing analysis of a sample of 52 rich clusters. The project involved developing  new accurate photometric procedures and  new algorithms for measuring cluster masses In ongoing work, this sample is being used to propose new cluster scaling relationships and to furnish new constraints on cosmological parameters .
\subsubsection{Detecting tidal stripping of halos using weak lensing}
A new approach to demonstrating the stripping of dark matter from the halos around distant group using weak gravitational lensing has been proposed. Its implementation was heavily simulated using the Millennium Simulation.
\subsubsection{Chandra observations of the largest quasar lens}
Blandford and Schrabback have participated in a investigation led by former postdoc Oguri to analyze Chandra observations of the largest separation, triply-imaged quasar, SDSS J 1029+2623. The X-ray observations from the associated z=0.58 rich cluster exhibit a subpeak suggestive of a merger and a mass consistent with that obtained from gravitational lensing. The consistency of the magnifications suggested that microlensing is not important in this system. However, their ratios point to the presence of dark matter substructure. A multi-telescope study of this massive X-ray cluster has been performed, finding additional multiple images, measuring a time delay and performing a weak lensing analysis. The lens model is able to account for the salient features of these observations, deriving a magnification of a background quasar of 30. There is a serious discrepancy with the X-ray cluster mass estimate which is attributed to shock heating in a merger. 
\subsection{Miscellaneous}
\subsubsection{Cosmic strings}
Morganson, Marshall, Schrabback and Blandford continue to explore ways to improve the lensing limit on the incidence of cosmic strings.
\subsubsection{Hubble workshop} 
Suyu, Blandford, Freedman and Treu, organized a highly successful workshop on the measurement of the Hubble constant. Many competing approaches were contrasted and compared.
\subsubsection{Angular Correlation Function}
Blandford is collaborating with Morganson and Schrabback on the measurement of cosmic shear using asymmetry in the angular correlation function on arc second angular scales following earlier work by Morganson and Blandford. The theory has been developed and an attempt is underway to seek this effect in existing data.
\subsubsection{Microlensing studies of nomad planets}
Blandford, Barnabe and Marshall have collaborated with former postdoc Strigari on an investigation of the use of microlensing observations, from ground and space, to enhance our statistical understanding of the frequency of planetary companions to main sequence stars in mass - semi major axis - eccentricity space in the light of the results from the Kepler space mission.  They have gone on  to assess the prospects of detecting interstellar planets and dwarf planets using microlensing. They argued that there could be as many as a hundred thousand of these �nomads� per star with masses larger than that of Pluto. In collaboration with Porter, they considered $\gamma$-ray limits on the incidence of nomads and began a study the probabilities of their carrying microbial life between stars (Panspermia). This last has led to a discussion of several experiments designed to quantify the viability of microorganisms under interstellar and re-entry conditions.
\end{document}
