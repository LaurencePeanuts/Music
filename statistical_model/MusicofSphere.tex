

\documentclass[prd, onecolumn, nofootinbib, floatfix]{revtex4-1}

\usepackage{amsmath}
\usepackage{graphicx}
\usepackage{dcolumn}
\usepackage{bm}
\usepackage{epsfig}
\usepackage{amssymb,latexsym,mathrsfs}
\usepackage{graphicx}
\usepackage{color}
\usepackage{hyperref}

\usepackage{tabu}

\hypersetup{
    colorlinks=true,
    linkcolor=red,
    citecolor=blue,
} 

% Making life easier
\newcommand{\be}{\begin{equation}}
\newcommand{\ee}{\end{equation}}
\newcommand{\bs}{\begin{split}} 
\newcommand{\bea}{\begin{eqnarray}}
\newcommand{\eea}{\end{eqnarray}}

\newcommand{\RdP}[1]{\textcolor{red}{[{\bf RdP}: #1]}}
 

% useful symbols
\newcommand{\om}{\Omega_m}
\newcommand{\ha}{\frac{1}{2}}
\newcommand{\ahub}{\frac{\dot{a}}{a}}
\newcommand{\ode}{\Omega_{de}}
\newcommand{\Oe}{\Omega_{e}}
\newcommand{\lcdm}{$\Lambda$CDM} 
\newcommand{\neff}{N_{\rm eff}} 
\newcommand{\hfid}{H^2_{\rm fid}} 
\newcommand{\dl}{\delta} 
\newcommand{\sumu}{\Sigma m_\nu} 
\newcommand{\mnu}{\Sigma m_\nu} 
\newcommand{\mpci}{\,{\rm Mpc}^{-1}}

\newcommand{\jcap}{JCAP}
\newcommand{\apjs}{Astrophys.J.Supp.}
\newcommand{\mnras}{MNRAS}
\newcommand{\physrep}{Physics Reports}

\begin{document}

\title{The Music of the Sphere} 
%\author{} 
%\affiliation{} 


\begin{abstract}


\end{abstract}

\date{\today} 

\maketitle

%%%%%%%%%%%%%%%%%%%%%%%%%%%%%%%%%%%%%%%%%%%%%%%%%%%%%%%%%%%%%%%%%%%%%%%%
\section{Generating constrained realizations}


I will jump, not always in a well-organized way, between the general notation from \cite{zaroubietal95}
and a more practical notation for the problem at hand (the simplext version of it).


Let's start with the following toy model.
We have a 3D potential field on a grid
\be
\Phi({\bf x}_i).
\ee
Here $i$ labels the grid points so that we can treat this field as a ``theory vector'',
\be
{\bf s} \equiv {\bf \Phi},
\ee
with components
\be
s_i = \Phi_i = \Phi({\bf x}_i) \quad i = 1,..,N.
\ee
Moreover, we have a $M$-dimensional data vector, given by some linear operator on the theory vector, plus noise,
\be
{\bf d} = {\bf R} \, {\bf s} + {\bf \epsilon}.
\ee
Here ${\bf R}$ is the transformation matrix defining the data, and the noise
has covariance
\be
N^{\epsilon}_{kl} \equiv \langle \epsilon_k \, \epsilon_l \rangle.
\ee
I will try to use the indices $i,j$ on the theory (i.e.~$\Phi$) side (so that these indices run from $1$ to $N$)
and use $k,l$ on the data side (so that they run from $1$ to $M$).
More concretely, let's say the data are simply measurements of $\Phi({\bf x}_i)$ at
a subset of specific
grid locations $\{i_1,...,i_M \}$,
\be
d_k \equiv \hat{\Phi}({\bf x}_{i_k}) = \Phi({\bf x}_{i_k}) + \epsilon_k = s_{i_k} + \epsilon_k.
\ee
Then, ${\bf R}$ is a $M \times N$ matrix with entries
\be
R_{ki} \equiv \delta^{K}_{i_k i}.
\ee
For simplicity, we can start with a diagonal noise matrix,
\be
N^{\epsilon}_{kl} = \sigma^2_\epsilon \, \delta^{K}_{k l}
\ee


Now, given a measurement of ${\bf d}$, i.e.~given a noisy measurement
of the potential at a subset of grid locations, for instance on a sphere,
we want to generate realizations of the full field. There is one more ingredient we need:
the prior covariance matrix of the theory vector.
This is a $N \times N$ matrix ${\bf S}$, with entries
\be
S_{ij} \equiv \langle s_i s_j \rangle = \langle \Phi({\bf x}_i) \, \Phi({\bf x}_j) \rangle = \xi_\Phi(|{\bf x}_i - {\bf x}_j|).
\ee

Now, following e.g.~\cite{rp92,zaroubietal95}, we can first construct the maximum likelihood estimator
for the field $\Phi({\bf x}_i)$.
\be
\label{eq:mv estimator}
{\bf s}^{\rm MV} = {\bf F} \, {\bf d},
\ee
with
\be
\label{eq:F matrix}
{\bf F} = \langle {\bf s} \, {\bf d}^\dagger \rangle \, \langle {\bf d} \, {\bf d}^\dagger \rangle^{-1} = {\bf S} \, {\bf R}^\dagger \left( {\bf R} \, {\bf S} \, {\bf R}^\dagger + {\bf N^\epsilon}  \right)^{-1}
\ee
Since we assume Gaussianity, this estimator gives the posterior expectation value for the field
and the maximum likelihood solution.

The above Eqs.~(\ref{eq:mv estimator})-(\ref{eq:F matrix}) give the general result. Things become clearer when we apply it to our specific example.
For instance,
\be
\langle d_k \, d_l \rangle = \left({\bf R} \, {\bf S} \, {\bf R}^\dagger  + {\bf N^\epsilon} \right)_{kl} = \xi_\Phi(|{\bf x}_{i_k} - {\bf x}_{i_l}|) + \sigma_{\epsilon}^2 \, \delta^K_{kl} \equiv \left({\bf \xi_\Phi} + \sigma^2_\epsilon \, {\bf I}\right)_{kl}
\ee
(${\bf I}$ is the identity matrix and ${\bf \xi_\Phi}$ is the correlation function matrix restricted to the grid points
where we have measurements, e.g. the CMB last-scattering surface) and
\be
\langle s_i \, d_k \rangle = \left( {\bf S} \, {\bf R}^\dagger \right)_{i k} = \xi_\Phi(|{\bf x}_{i} - {\bf x}_{i_k}|).
\ee
We can't proceed further analytically, because we need to invert the matrix ${\bf \xi_\Phi} + \sigma^2_\epsilon \, {\bf I}$,
which is simply the total covariance matrix of the observable $d_k \equiv \hat{\Phi}({\bf x}_{i_k})$.
We can now write for the solution,
\be
\label{eq:mv sol}
s^{\rm MV}_i = \Phi^{\rm MV}({\bf x}_i) = \sum_{kl}  \xi_\Phi(|{\bf x}_{i} - {\bf x}_{i_k}|) \, \left({\bf \xi_\Phi} + \sigma^2_\epsilon \, {\bf I}\right)^{-1}_{kl} \, \hat{\Phi}({\bf x}_l)
\ee


%S_{ij} = xi_ij
%R=MxM unity + zeros, (MxN)
%Rdag = MxM unity + zeros (NxM)
%Neps = unity (MxM)
%RSRdag = xi_kl,

%in parentheses:

%xi_{kl}

%S Rdag = xi_ik

%sherman-morrison
%eps = A - M b^T
%epsinv = ...

%in fourier space,
%S diagonal, either R not or N complicated,

\subsection{Generating realizations of $\Phi({\bf x}_i)$}

Now that we have the {\it mean} of the conditional posterior of ${\bf s} = {\bf \Phi}$, given in Eq.~(\ref{eq:mv sol}), there is a straightforward
method for generating realizations that follow the posterior distribution:
\begin{itemize}
\item
Generate an unconstrained Gaussian random realization of the theory field, ${\bf \tilde{s}} = {\bf \tilde{\Phi}}$ (i.e.~the 3D potential), that follows the prior
correlation function of the field, i.e.~the same way of generating random realizations we use for the true field, that is our
input to the whole calculation.
\item
Generate a (Gaussian) realization of the measurement noise ${\bf \epsilon}$, that follows the covariance
given by ${\bf N^{\epsilon}}$.
\item
Construct a realization of the data:
\be
{\bf \tilde{d}} = {\bf R} \, {\bf \tilde{s}} + {\bf \epsilon}.
\ee
Specifically, for us this means
\be
\tilde{\hat{\Phi}}({\bf x}_{i_k}) = \tilde{\Phi}({\bf x}_{i_k}) + \tilde{\epsilon}_k.
\ee
\item
Write the realization of the field, {\it conditional on the observations}, as
\be
{\bf s}_{\rm post} = {\bf s}^{\rm MV} + {\bf \tilde{s}} - {\bf F} \, {\bf \tilde{d}}.
\ee
For us this means
\be
\Phi_{\rm post}({\bf x}_i) = \Phi^{\rm MV}({\bf x}_i) + \tilde{\Phi}({\bf x}_i) - \tilde{\Phi}^{\rm MV}({\bf x}_i),
\ee
where I have written
\be
\tilde{\Phi}^{\rm MV}({\bf x}_i) \equiv \left({\bf F} \, {\bf \tilde{d}}\right)_i = \sum_{kl}  \xi_\Phi(|{\bf x}_{i} - {\bf x}_{i_k}|) \, \left({\bf \xi_\Phi} + \sigma^2_\epsilon \, {\bf I}\right)^{-1}_{kl} \, \tilde{\hat{\Phi}}({\bf x}_l)
\ee
This quantity has the appropriate statistics given the measurement ${\bf d}$ and
the prior statistics of ${\bf s}$ and of the noise.
\end{itemize}

Possibly more advanced method that would be interesting to explore: \cite{jaslav15}.

\bibliography{refs}

\end{document}

