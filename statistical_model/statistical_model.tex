%
%  statistical_model.tex
%
%  Created by Michael D. Schneider on 2015-01-13
%    <schneider@ucdavis.edu>
%
\documentclass[11pt, letterpaper]{article}

\usepackage[OT1]{fontenc}
\usepackage[in]{fullpage}
\usepackage{natbib}

\usepackage[,bookmarks,bookmarksopen,colorlinks,linkcolor={black},citecolor={black},urlcolor={red}]{hyperref}
\usepackage{graphicx}
\usepackage{bm}
\usepackage{amsmath,amssymb}
\usepackage{aas_macros}

\usepackage{mathptmx}  % times roman, including math (where possible)

% Macros
\newcommand{\half}{\frac{1}{2}}
\newcommand{\rhocrit}{\rho_{\rm crit}}
\newcommand{\rvir}{r_{\rm vir}}
\newcommand{\mvir}{m_{\rm vir}}
\newcommand{\om}{\Omega_{m}} 
\newcommand{\kv}{\mathbf{k}}
\newcommand{\xv}{\mathbf{x}}
\newcommand{\mv}{\mathbf{m}}
\newcommand{\muv}{\bm \mu}
\newcommand{\hmsun}{h^{-1}M_{\odot}}
\newcommand{\hmpc}{h^{-1}Mpc}
\newcommand{\hgpc}{h^{-1}{\rm Gpc}}

%%% "Data" - i.e., the observed CMB temperature map
\newcommand{\data}{\mathbf{d}}
%%% "Signal" - i.e., zero-noise CMB temperature
\newcommand{\signal}{\mathbf{s}}
%%% "noise" - i.e., the pixel noise realization
\newcommand{\noise}{\mathbf{n}}
%%% Transfer function relating the 3D gravitational potential to the
%%% 2D CMB temperature (or other) map
\newcommand{\transfer}{\mathsf{A}}
%%% Signal and noise covariance matrices
\newcommand{\Smat}{\mathsf{S}}
\newcommand{\Nmat}{\mathsf{N}}
\newcommand{\Psimat}{\mathsf{\Psi}}
\newcommand{\Sigmamat}{\mathsf{\Sigma}}
\newcommand{\Cmat}{\mathsf{C}}
%%% Gravitational potential represented as a vector of "voxels" or similar
\newcommand{\gravpot}{\bm \Psi}
%%% Normal (Gaussian) distribution
\newcommand{\normdist}{\mathcal{N}}

% PROBABILITY THEORY
\def\pr{{\rm Pr}}

%%%%%%%%%%%%%%%%%%%%%%%%%%%%%%%%%%%%%%%%%%%%%%%%%%%%%%%%%%%%%%%%%%%%%%%%%%%%%%%
\begin{document}
  
\title{Statistical model for CtU hack: `The Music of the Sphere'}

\author{CtU hack team}

\date{\today}

\maketitle
% \tableofcontents


Our goal is to draw posterior samples of the large-scale 3D cosmological mass 
density (or gravitational potential) in our universe given a map of the CMB. 

% ---------------------------------------------------------
\section{Data model}
The observed CMB temperature $\data$ is modeled as a sum of a signal $\signal$
and $\noise$, neglecting foregrounds,
\begin{equation}
	\data = \signal + \noise,
\end{equation}
where both $\signal$ and $\noise$ are Gaussian distributed with zero means 
and covariances $\Smat$ and $\Nmat$.

% ---------------------------------------------------------
\section{3D potential - relation to data}
We want to sample realizations of the 3D gravitational potential $\gravpot$
given the observed 2D CMB map $\data$. We can model the relationship between 
the potential and the CMB temperature as a linear transfer function matrix 
$\transfer$,
\begin{equation}
	\signal \equiv \transfer \gravpot.
\end{equation}


% ---------------------------------------------------------
\section{Likelihood}
The likelihood is a Gaussian distribution given our assumption of a 
Gaussian noise model,
\begin{equation}
	\pr(\data | \signal, \Nmat) = \normdist_{\data} 
	\left(\signal, \Nmat\right)
	= \normdist_{\data} \left(\transfer \gravpot, \Nmat\right)
\end{equation}

The prior on the gravitational potential is,
\begin{equation}
	\pr(\gravpot | \Psimat) = 
	\normdist_{\gravpot} \left(0, \Psimat\right)
\end{equation}

% ---------------------------------------------------------
\section{Conditional posteriors for Gibbs sampling}
\citet{Wandelt:2003uk} shows that the conditional distribution 
of the signal or gravitational potential is a Gaussian with a 
mean and covariances (also known as the `Wiener filter'),
\begin{equation}\label{eq:psi_cond_post}
	\pr(\gravpot | \data, \Psimat, \Nmat) 
	= \normdist_{\gravpot}
	\left(\muv, \Sigmamat\right),
\end{equation}
where,
\begin{align}
	\muv &\equiv \Psimat 
	\left(\Psimat + \Cmat_{N}\right)^{-1}\mv
	\label{eq:psi_cond_mean}
	\\
	\Sigmamat &\equiv 
	\Psimat 
	\left(\Psimat + \Cmat_{N}\right)^{-1} \Cmat_{N},
	\label{eq:psi_cond_cov}
\end{align}
\begin{align}
	\mv &\equiv	
	\left(\transfer^{T} \Nmat^{-1}\transfer\right)^{-1}
	\transfer^{T}\Nmat^{-1} \data 
\end{align}
is the maximum likelihood estimator of the signal and,
\begin{equation}
	\Cmat_{N} \equiv \left(\transfer^{T} \Nmat^{-1}\transfer\right)^{-1}.
\end{equation}

% ---------------------------------------------------------
\section{Summary for code implementation}
Our goal is to sample conditional Gaussian draws from \autoref{eq:psi_cond_post}. 
An important point is that the vector $\muv$ can be extended from just those 
voxels that encompass the last scattering surface to also those voxels 
filling the rest of the universe. Then, posterior inference of the mass density 
away from the last scattering surface is described by a conditional 
multivariate Gaussian distribution with covariance matrix given by the 
Schur complement of $\Sigmamat$.

The chief computational challenge is inverting the matrices in 
\autoref{eq:psi_cond_mean} and \autoref{eq:psi_cond_cov}.
\citet{Wandelt:2003uk} and \citet{Wandelt:2004ul} describe 
a conjugate gradient based method to evaluate the Wiener filter terms for 
Gibbs sampling. 

A secondary computational issue is specification of the discretized 
matrix $\transfer$ which represents the transfer function relating the 
3D graviational potential to the 2D CMB anisotropies.


% We an probably start with non-optimized matrix inverses in a suitably 
% scaled toy model and expand to larger numbers of pixels as time 
% and interest permit. 

% ---------------------------------------------------------
\bibliographystyle{apj}
\bibliography{music}


\end{document}

